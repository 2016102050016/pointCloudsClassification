\subsection{Compte-rendu de la réunion du lundi 9 mars}

\subsubsection{Travail sur les spin images}

La spin image est l’image du nuage 3D prise en un point, perpendiculairement à la normale en ce point (donc selon le plan tangent).
Le but est de prendre est de prendre les meilleures spin images pour chaque nuage de point car c’est sur ces images 2D que nous allons réaliser la classification. Il est compliqué de trouver la meilleure spin image. Dans un premier temps il faut trouver les meilleures spin images sur le vélo manuellement. Dans un second temps on pourra l’automatiser en se servant de la bounding box. Attention cependant, si l’objet est tourné par rapport aux axes (n’est pas parallèle aux axes) la bounding box sera faussée. Il faut alors se servir du centre selon les différents axes mais ces calculs sont compliqués.
La classe SpinImage de PCL peut nous être utile, il faudra notamment se servir des fonctions setRotation() et setInputCloud()

\subsubsection{ Classification}

Il faut découvrir les différentes fonctionnalités de PCL pour faire de la classification. A priori il n’y a que des fonctions de clusterings, c'est-à-dire de l'apprentissage non-supervisé. On peut utiliser ces fonctions dans un premier temps. Il faudra ensuite faire de l'apprentissage supervisé, en modifiant par exemple le code source des fonctions. Il faudra pour cela télécharger le code source de PCL, le modifier et le compiler.

\subsubsection{ROS}

ROS sert à traiter les .tm, qui sont des vidéos en 3D, pour en extraire les .pcd. C’est un système d’exploitation, il faut le lancer avec roscore \& puis executer le C++. Vu qu’il y a des erreurs dans la compilation du C++, probablement liées à ROS, M. Guerrero nous fournira directement les archives contenant les pcd. Nous possédons  pour l’instant une séquence de nuage de points correspondant aux images 3D d’une vidéo d’un seul vélo en 3D, nous pouvons donc commencer par travailler sur ces données en attendant que le problème de ROS soit résolu.