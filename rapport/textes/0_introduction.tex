Aujourd'hui, les technologies et l'intelligence artificielle ne cessent de progresser. Les appareils que nous utilisons dans la vie de tous les jours, tels que les téléphones portables et appareils électroménagers, sont de plus en plus performants mais aussi plus intelligents. C'est aussi le cas de nos véhicules de transport, comme les voitures qui peuvent se garer automatiquement, et parfois même détecter des "objets" tels que des piétons.\\

Les voitures grand public sont donc en grande majorité équipées de différents capteurs. Les capteurs CMOS (complementary metal oxide semiconductor) et CCD (charge coupled device) sont particulièrement utilisés pour ces applications. Les CCD, créés en 1969, sont composés d'une série de condensateur semi-conducteurs chargés par la lumière selon son intensité. Les charges sont ensuite transformées en tension pour être mesurer. Ces capteurs, plus utilisés à l'origine, étaient moins sensibles au bruit et avaient une sortie de bonne qualité. Ils ont petit à petit été remplacés par les capteurs CMOS, créés au début des années 90. Ces capteurs sont moins chers et plus simples à utiliser, ce qui explique leur succès. Ces capteurs sont utilisés sur beaucoup de voitures, par exemple pour aider à la conduite de nuit, ou surveiller la somnolence du conducteur.\\

Les voitures orientées recherche ont davantage d'intelligence et de précision ; dès 2005, la voiture "Alice" de la Team Caltech pouvait rouler 300 miles en autonomie et en évitant les obstacles en un environnement inconnu, dans le cadre du DARPA Grand Challenge dans le désert des Mojaves. La voiture "Junior", de Stanford University, est capable de naviguer dans un environnement urbain en autonomie. Elle est capable de sélectionner ses propres routes, percevoir le traffic alentour, et effectuer des manœuvres telles que des changements de voies, des demi-tours et des stationnements. Elle a ainsi obtenu la seconde place du DARPA Urban Challenge de 2007 qui se déroulait cette fois en milieu urbain. La plateforme PACPUS du laboratoire Heudiasyc, rattachée à l'Université de Technologie de Compiègne, est composée de plusieurs véhicules expérimentaux qui permettent de tester des aides à la conduite automobile et des fonctions ADAS (Advanced Driver Assistance Systems). Ces véhicules sont équipés de capteurs et logiciels permettant des applications comme la localisation précise (trajectométrie de précision centimétrique), la perception de l'environnement et du conducteur grâce à des télémètres laser, caméras numériques et radars, ou encore des mesures temps réel de la dynamique du véhicule (dérive, forces sur les roues).\\

Justement, pouvoir détecter des objets susceptibles d'entrer en collision avec son véhicule est un progrès considérable pour la sécurité de chacun. Savoir de plus quel est cet objet, afin d'adopter un comportement propre à chacun, est encore plus intéressant. Nous pouvons par exemple demander au véhicule de ralentir lorsqu'un cycliste ou un piéton est à proximité, et continuer sa route si l'objet identifié est un arbre, puisque celui-ci ne peut a priori pas bouger.\\

Nous pouvons dès lors nous demander comment une telle classification d'objets pourrait être réalisée ; comment capturer des données et les utiliser à bon escient afin d'en extraire des caractéristiques propres à chaque objet ?\\

Nous allons donc étudier dans un premier temps les attributs nous permettant de réaliser la classification. Dans un deuxième temps, nous allons décrire les classifieurs utilisés après avoir extraits les attributs propres à chaque classe d'objet. Enfin, nous allons parler de la librairie C++ PointClouds, qui peut être intéressante pour traiter des images 3D.\\