Aujourd'hui, les technologies et l'intelligence artificielle ne cessent de progresser. Les appareils que nous utilisons dans la vie de tous les jours, tels que les téléphones portables et appareils électroménagers, sont de plus en plus performants mais aussi plus intelligents. C'est aussi le cas de nos véhicules de transport, comme les voitures qui peuvent se garer automatiquement, et parfois même détecter des \enquote{objets} tels que des piétons.\\

Justement, pouvoir détecter des objets susceptibles d'entrer en collision avec son véhicule est un progrès considérable pour la sécurité de chacun. Savoir de plus quel est cet objet, afin d'adopter un comportement propre à chacun, est encore plus intéressant. Nous pouvons par exemple demander au véhicule de ralentir lorsqu'un cycliste ou un piéton est à proximité, et continuer sa route si l'objet identifié est un arbre, puisque celui-ci ne peut a priori pas bouger.\\

Nous pouvons dès lors nous demander comment une telle classification d'objets pourrait être réalisée ; comment capturer des données et les utiliser à bon escient afin d'en extraire des caractéristiques propres à chaque objet ?\\

Nous allons donc étudier dans un premier temps les attributs nous permettant de réaliser la classification. Dans un deuxième temps, nous allons décrire les classifieurs utilisés après avoir extraits les attributs propres à chaque classe d'objet. Enfin, nous allons parler de la librairie C++ PointClouds, qui peut être intéressante pour traiter des images 3D.\\