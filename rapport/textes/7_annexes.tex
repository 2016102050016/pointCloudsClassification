Notre dossier principal est composé de 4 dossiers :

\dirtree{%
.1 /.
.2 dataset.
.2 matlab.
.2 pointclouds.
.2 rapport.
.2 track\_tool.
}

\vspace{10pt}

Le dossier \textit{dataset} est composé de différents jeux de données au format CSV que nous utilisons pour notre classification. D'autres fichiers non utilisés pour la classification se trouvent dans le dossier \textit{unused}. \\

\dirtree{%
.1 /dataset.
.2 dish\_area\_dataset.
.3 attributes.csv.
.3 attributesSmall.csv.
.3 intensityAttributes.csv.
.3 labels.csv.
.3 labelsSmall.csv.
.2 lomita.
.3 attributesSmall\_without\_unlabeled.csv.
.3 attributes\_without\_unlabeled\_and\_reduced\_background.csv.
.3 labelsSmall\_without\_unlabeled.csv.
.3 labels\_without\_unlabeled\_and\_reduced\_background.csv.
.3 unused.
}

\vspace{10pt}

Le dossier \textit{matlab} contient nos algorithmes Matlab pour la classification. Le dossier \textit{dish\_area} contient les SVM one versus all et one versus one ainsi que les K-moyennes sur le dataset dish area à 2 classes. Le dossier \textit{lomita} contient les SVM one versus all et one versus one ainsi que les K-moyennes sur le dataset lomita multi-classes. Il ne faut pas oublier d'ajouter le dossier \textit{resources} au path de Matlab puisqu'il contient la toolbox développée par M. Canu et splitdata.m. \\

\dirtree{%
.1 /matlab.
.2 boundingbox.
.3 loadpcd.m.
.3 minboundbox.m.
.3 mon\_bounding\_box.m \DTcomment{à exécuter}.
.3 plotminbox.m.
.2 dish\_area.
.3 Kmoyennes.
.4 Kmoyennes.m \DTcomment{à exécuter}.
.3 svm\_gaussian.
.4 svm\_error.m.
.4 svm\_kernel.m.
.4 svm\_predict.m.
.4 svm\_train.m.
.4 testSVM.m \DTcomment{à exécuter}.
.3 svm\_linear.
.4 svm\_error.m.
.4 svm\_predict\_linear.m.
.4 svm\_train\_linear.m.
.4 testLinearSVMSmall.m \DTcomment{à exécuter}.
.2 lomita.
.3 svm\_OneVsAll.
.4 svm\_one\_vs\_all.m \DTcomment{à exécuter}.
.3 svm\_OneVsOne.
.4 svm\_error.m.
.4 SVM\_OneVsOne.m \DTcomment{à exécuter}.
.4 svm\_predict\_linear.m.
.4 svm\_train\_linear.m.
.3 svm\_OneVsOne\_simplified.
.4 svm\_error.m.
.4 SVM\_OneVsOne.m \DTcomment{à exécuter}.
.4 svm\_predict\_linear.m.
.4 svm\_train\_linear.m.
.3 expend\_labels.m.
.3 get\_labels.m.
.3 Kmoyennes.
.4 Kmoyennes.m \DTcomment{à exécuter}.
.3 reduce\_background\_class.m.
.3 reduce\_background\_class\_small.m.
.2 pre\_processing.
.3 attributes.m.
.3 diminutionDonnees.m.
.3 loadpcd.m.
.3 pclviewer.m.
.3 rotate.m.
.3 savepcd.m.
.3 stats.m.
.2 resources.
.3 SVM-KM.
.3 splitdata.m.
}

\vspace{10pt}

Le dossier \textit{pointclouds} contient différents essais que nous avons effectués avec la librairie C++ Pointclouds. \\

\dirtree{%
.1 /pointclouds.
.2 essai\_compilation.
.2 essai\_segmentation.
.2 essai\_segmentation\_2.
.2 lecture\_data.
.2 spin\_image.
.2 spin\_image\_sur\_cylindre.
.2 visualisation.
.2 visualizer.
.2 visualizer\_2.
}

\vspace{10pt}

Le dossier \textit{track\_tool} contient l'outil permettant de transformer la vidéo au format ts en nuages de points au format pcd.\\

le dossier \textit{rapport} contient notre rapport au format PDF ainsi que ses fichiers sources.\\
