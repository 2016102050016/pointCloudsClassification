Finalement, nous avons pu constater que la classification d'objets, aujourd'hui de plus en plus présente notamment dans les véhicules intelligents, n'est pas si compliquée que cela. En effet, en extrayant des nuages de points à partir de vidéos, puis en calculant de simples attributs sur ces nuages, nous avons pu classifier des objets tels que des piétons ou des arbres dans leurs classes respectives. Bien évidemment, pour améliorer encore nos résultats, nous pourrions ajouter beaucoup d'autres attributs tels que des spin images ou des bounding boxes exactes par exemple, en entrée dans notre classifieur. Nous pourrions aussi améliorer les choix des hyper-paramètres dans nos SVM (Support Vector Machines). Enfin, nous pourrions essayer d'autres classifieurs ; les forêts aléatoires (Random Forests), notamment, ou des techniques de boosting comme Adaboost pourraient être une piste de comparaison.\\

Même si nous avons perdu du temps à essayer de manipuler la librairie Pointclouds, nous avons pu apprendre un peu à la manipuler, et nous la connaissons désormais un peu afin éventuellement de l'utiliser dans d'autres projets plus adéquats par rapport aux capacités de cette librairie.\\

Enfin, nous sommes satisfaits d'avoir pu appliquer nos connaissances théoriques d'apprentissage automatique sur un problème concret. Comme dit précédemment, la classification d'objets va se rendre de plus en plus indispensable dans des dispositifs qui ont besoin d'informations en temps réel, et de réagir en conséquence ; grâce à cela, peut-être aurons nous bientôt des robots faisant la cuisine en choisissant les ingrédients grâce à des caméras?\\